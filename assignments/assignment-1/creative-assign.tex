\documentclass[12pt, a4paper]{article}
\usepackage{setspace}
\doublespacing
\usepackage[backend=biber, style=mla, citestyle=mla, sorting=anyt]{biblatex}
\addbibresource{references.bib}
% \usepackage{times}
\usepackage{geometry}
\geometry{a4paper, margin=1in}
\author{Robby Renz \textsc{Abeysinghe}}
\title{An Interview with Thetis}

\begin{document}

\maketitle
\begin{center}
\begin{tabular}{l r}
Professor: & Dr. Conor Whately \\
Course Name: & Classical Mythology \\
Course Number: & CLAS-2701-01
\end{tabular}
\end{center}
\newpage

\section{Introduction}
I was given the extraordinary honour to get an exclusive interview with Thetis, the sea-nymph, of Greece (\cite{homer_iliad_2015} 18). In the next section is the script of the interview. Some words have been slightly edited for brevity's sake, but the essence of the interview has been preserved. 

\section{The Script}

\textbf{Robby}: Please state your name and a really brief introduction about yourself for the record.

\textbf{Thetis}: My Name is Thetis, and I am a sea nymph (\cite{homer_iliad_2015}). I was courted by the Gods Poseidon and Zeus, but when Themis prophesized that I would give birth to a male child that would eventually be more powerful than the father, I was married off to Peleus (\cite{wrath-of-thetis} 12), and together, we bore a son named Achilles (\cite{wrath-of-thetis} 1).

\textbf{Robby}: Thank you for that, and thank you so much for agreeing to do this with me. I really appreciate it. And I am so sorry about the death of your son. I read about him and I know that he was a very brave man who would die for the people that means so much to him.

\textbf{Thetis}: Thank you.

\textbf{Robby}: I would like to talk to you about the Trojan War, Achilles, and your role in it. 

\textbf{Thetis}: Go right ahead.

\textbf{Robby}: Why do you think that, after Agamemnon took Briseis, that Achilles decided to plead to you to seek a favour from Zeus, and not from other, more powerful Gods that surely has his ear, like Hera or Athena (\cite{wrath-of-thetis} 1-2)?

\textbf{Thetis}: He told me that he heard that I have saved Zeus, all by myself, from being binded (\cite{wrath-of-thetis} 10). So he felt that he owed me a favour. Also, if I might add, I believe that another reason as to why Zeus kept his promise to me is that he is still in love with me. So those two factors helped play a role in his cooperation (\cite{wrath-of-thetis} 12).

\textbf{Robby}: Okay, next question; what did you feel about when he decided not to participate in the Trojan War because of what Agamemnon took from him?

\textbf{Thetis}: I was ectastic at first, because that means he would not die, but then, it hit me that it was his destiny to die in the war, so it did not matter anyway. However, there was a little part of me that wanted to believe that he can survive this war, so I kept holding on to that feeling of hope. Furthermore, I personally believe that Achilles did not want to fight, not because he believed the oracle, but because of the pain he feels, the pain of being humiliated by someone who is much higher in power than he is, and he is unable to do anything about it. To me, I do not care if he did not believe his destiny, I am just glad that he is not joining his brothers-in-arms in battle, even if his reasons has nothing to do with the oracle (\cite{homer_iliad_2015} Book 16 1-100).

\textbf{Robby}: Speaking about his destiny, how does it make you feel knowing that no matter what you do, you, most likely, cannot save Achilles?

\textbf{Thetis}: It was infuriating; I was full of sorrow and overcome with grief, even before my dear son's death. I tried to warn him repeatedly, but he would not listen. I even went as far as to seek Hephaestus' help because I believed in a small fragment of myself that he can be saved (\cite{wrath-of-thetis} 8).

\textbf{Robby}: When Achilles finally kills Hector out of vengenance (\cite{homer_iliad_2015}), what was going through your mind, knowing that his death is about to happen?

\textbf{Thetis}: I was overcome with grief, obviously, because now I know that he is nearing his time. And if I can be honest, I thought that if anybody has a chance of killing Achilles, it would be Memnon. I was extremely terrified when they faced off.

\textbf{Robby}: Why would you think that Memnon had a good chance of kiing Achilles?

\textbf{Thetis}: Because think about it, Memnon and Achilles are two sides of the same coin. Memnon's mother is that of a divine, Eos, and his father is a mortal. That fact alone means he has a lot in common with Achilles (\cite{wrath-of-thetis} 3). I still remember the time when I was standing just in front of Eos while our sons were about to do battle. She was on the other side, and both of us were praying to Zeus for the safety of our sons (\cite{aethiopus} 345). That final showdown was really anybody's game, but in the end, we all know how it ended.

\textbf{Robby}: Well, that is all I have for you today. Thank you so much for doing this again. Is there anything you would like to add before we call it a day?

\textbf{Thetis}: All of what happened is ironic, is it not?

\textbf{Robby}: What do you mean by that?

\textbf{Thetis}: I managed to save the Gods Hephaestus and Dionysus. I've managed to save even Zeus himself (\cite{wrath-of-thetis} 10), but after all that I have done, I could not even save my own son. Because of what has transpired, I have never felt so helpless in my entire life than what I felt during the Trojan War. I could have done more, but my dear Achilles's destiny has already been written in the stars. I do not have to accept his death, I can just move on with my life (\cite{wrath-of-thetis} 1-2).

\textbf{Robby}: I am so sorry for your loss again, Thetis.

\begin{center}
	\textbf{End of Transcript}
\end{center}

% \medskip
\newpage

\printbibliography
\end{document}

