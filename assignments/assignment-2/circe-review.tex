\documentclass[12pt, a4paper]{article}
\usepackage{setspace}
\doublespacing
\usepackage[backend=biber, style=mla, citestyle=mla, sorting=anyt]{biblatex}
\addbibresource{references.bib}
% \usepackage{times}
\usepackage{geometry}
\geometry{a4paper, margin=1in}
\author{Robby Renz \textsc{Abeysinghe}}
\title{A Review of Madeline Miller's Circe}

\begin{document}

% start of title page
\maketitle
\begin{center}
\begin{tabular}{l r}
Professor: & Dr. Conor Whately \\
Course Name: & Classical Mythology \\
Course Number: & CLAS-2701-01
\end{tabular}
\end{center}
\newpage
% end of title page

Do not forget to check if the references are correct (\cite{homer_odyssey_1998}), like check if you are using the right editions, year it was published etc. (\cite{miller_circe_2018}).

\begin{itemize}
	\item I liked how they humanized Circe; how? By detailing how she felt with Glaucos, Prometheus etc. and compare it with her appearance in Circe
	\item I liked reading how Helios and Perse thought and regarded humans.
	\item I loved the interactions between Prometheus and Circe, because it humanizes her.
	\item I liked seeing how the Titans and the Olympians kept the peace.
	\item it was very interesting seeing Helio's attitudes and his life in general, what he thought about mortals
	\item Also, talk about Daedulus and the brief appearance of his son, Icarus.
	\item Also, state how having Hermes state the story of how Icarus died, and having it stated so briefly in like one paragraph, rather than having it spread out in three to four paragraphs or dedicate an entire chapter to it gives Icarus death much more shocking death???
	\item I just found out that Pasiphae had sex with the white bull because she was cursed by Poseidon, according to fiction, but in the book she willingly wanted to have sex with the bull because she wanted to be known (was she also jealous?). State that this was an interesting change in story.
	\item state that Daedulus and Pasiphae were standouts in my opinion.
	\item But also dedicate at least 85\% of the paper to the characterization of Circe, she is the star after all.
\end{itemize}

This article is about a book called Circe, a fictional novel by Madeline Miller. Note that this paper will not be a book review but rather, it is going to be a verdict on the novel; I am going to state what I really liked about the novel, and elaborate on it.  But long story short, I really loved this book; it kept me engaged until the very end.

Another point I would like to add is that I loved the way Miss Miller introduced Daedelus 

In conclusion, as I have said before, I truly enjoyed every second of this book. It was such a page-turner thanks to Miss Miller's amazing ability to introduce characters and give them a three-dimensional overview of them. That alone is what I feel made the book such a winner, in my opinion. It had a lot of heart and it was emotional, Despite the fact that the story was not original, as she adapted it from ancient sources, it still had an amazing story, thanks to the fact that she was able to make virtually every single character in th Sure, the story was not original as the she adapted it from I give it 10/10.I loved how she took fictional characters and gave them heart and character, and really made it feel that they are actual people, even though some of them are Gods or Titans.

\newpage
\printbibliography
\end{document}

