\documentclass[12pt, a4paper]{article}
\usepackage{setspace}
\doublespacing
\usepackage[backend=biber, style=mla, citestyle=mla, sorting=anyt]{biblatex}
\addbibresource{references.bib}
% \usepackage{times}
\usepackage{geometry}
\geometry{a4paper, margin=1in}
\author{Robby Renz \textsc{Abeysinghe}}
\title{A Review of Madeline Miller's Circe}

\begin{document}

% start of title page
\maketitle
\begin{center}
\begin{tabular}{l r}
Professor: & Dr. Conor Whately \\
Course Name: & Classical Mythology \\
Course Number: & CLAS-2701-01
\end{tabular}
\end{center}
\newpage
% end of title page

This short paper is all about a book called Circe, which is a fictional novel by Madeline Miller. Note that following paragraphs will not be a review about the book, but rather, it is going to be a verdict on the novel; I am going to write about what I really liked about the novel, and elaborate on it. Futhermore, I would also state my opinion about the difference between the characters that appear in ancient sources versus her retelling of it. Note that whenever I refer to historical or ancient references, I always refer to sources from the website theoi.com (\cite{theoi}). But long story short, I am genuinely fond of this novel; it kept me engaged until the very end. 

To start off, I would like to state what I truly appreciated in Miller's novel. Firstly, I loved how she was able to illustrate the uneasiness in the pact between the Titans and the Olympians (\cite{miller_circe_2018} 12 and 62), because it gave me a first person glance as to how the Titans and the Olympians felt about living in a time after the Titanomachy. Secondly, I was impressed how Miller presented Helios. He had a temper (\cite{miller_circe_2018} 3), enjoyed playing checkers (\cite{miller_circe_2018} 4) and looked down upon mortals (\cite{miller_circe_2018} 3). I also enjoyed reading about Circe's time spent with Helios, at first the author presented the interaction as a tender father spending time with one of his daughters like the fact that he let her ride in his chariot and let her see his herd of cattle (\cite{miller_circe_2018} 6), but as I read further into the book, he was just not at all caring. Ultimately, its the little things, the little attention to detail, that give the characters a breath of fresh air and in turn, give this book a charming flavour. Personally, the outstanding character development alone is what I feel made the book such a masterpiece.

Now onto the star of the novel, Circe. The Circe that I was introduced in Miller's novel was a far cry from the Circe that I encountered in Homer's The Odyssey (\cite{homer_odyssey_1998} 88-92). I felt that Circe in the Odyssey played a small role, but it was an important one none the less. In the Odyssey, I thought that Circe was presented as just some generic witch that ultimately helps Odysseus to head home. There was no any character development of her. Granted, The Odyssey was not about Circe, but even for her important role, there was not much to make of her. However, in Miller's Circe, that is a whole another story. For starters, Miller made a great move in introducing Prometheus (\cite{miller_circe_2018} 11) early on in the novel, because I feel that was when it was a life-changing moment for her because it was clearly a traumatic moment. Plus, that scene clearly showed that she had empathy for mortals (\cite{miller_circe_2018} 15-18), unlike the rest of her kind. In addition, her interactions with . All these interactions help strengthen Circe as a character and really made me feel for her, because all those events humanized Circe and made her a believable witch, if there ever was one.

Another point I would like to add is that the two other standout characters, apart from Circe, are Daedalus and Pasiphae, in my opinion. First off, for Pasiphae, she was introduced very early on in the novel (\cite{miller_circe_2018} 5). She was shown to be an extremely unpleasant witch, always criticizing Circe every chance she gets, and with an absolute disregard for human life, such as how she treats Daedalus and the soldiers she sent to retrieve Circe (\cite{miller_circe_2018} 91). It was with Pasiphae that I noticed that Miller took certain liberties at changing the historical material to suit the plot of the novel. According to ancient sources, Pasiphae had sex with the bull because Poseidon cursed her. But in Circe, Pasiphae did the deed as she wanted to be famous and relevant again (\cite{miller_circe_2018} 117). I found this an interesting discrepancy, but it is one that I found to be quite helpful in pushing the plot forward in terms of her relationship with Pasiphae as well as her relationship with Daedalus.

For Daedalus too was introduced very early on in the novel(\cite{miller_circe_2018} 28). The conversation that occurred between Aeetes and Circe when he points Daedalus out to Circe at Pasiphae's wedding (\cite{miller_circe_2018} 28) was an indication to the reader that he is going to have an impact in Circe's life somehow. He was shown to be resilient, calm and generous. Furthermore, I loved the short scene with him and his son, Icarus (\cite{miller_circe_2018} 122). People who are aware of who Icarus is and how he died would find that scene all the more touching, so I am glad that Miller decided to go that route. Also, that scene alone gives the exposition of how Icarus died (\cite{miller_circe_2018} 131) all the more shock value, even for readers who know of Icarus' fate.

In conclusion, as I have said before, I honestly enjoyed every second of this book. It was such a page-turner thanks to Miller's amazing ability to introduce and develop characters and give them a three-dimensional makeover of them and make them feel lifelike. Taking fictional characters and giving them heart making you feel for them is no small feat, considering the fact that some of the people there are Gods or Titans, making it hard to relate to them, but Miller was able to do just that.  It had a lot of heart and it was emotional when it had to be. Despite the fact that the story was not original, as she adapted it from ancient and historical sources, the way she weaved the plot and presented was exceptional. I recommend this book to anybody wanting to read their next favourite book.

\newpage
\printbibliography
\end{document}
