\documentclass[12pt, a4paper]{article}
\usepackage{setspace}
\doublespacing
\usepackage[backend=biber, style=mla, citestyle=mla, sorting=anyt]{biblatex}
\addbibresource{references.bib}
% \usepackage{times}
\usepackage{geometry}
\geometry{a4paper, margin=1in}
\author{Robby Renz \textsc{Abeysinghe}}
\title{A Review of Madeline Miller's Circe}

\begin{document}

% start of title page
\maketitle
\begin{center}
\begin{tabular}{l r}
Professor: & Dr. Conor Whately \\
Course Name: & Classical Mythology \\
Course Number: & CLAS-2701-01
\end{tabular}
\end{center}
\newpage
% end of title page

\begin{itemize}
	\item Also, talk about Daedulus and the brief appearance of his son, Icarus.
	\item Also, state how having Hermes state the story of how Icarus died, and having it stated so briefly in like one paragraph, rather than having it spread out in three to four paragraphs or dedicate an entire chapter to it gives Icarus death much more shocking death???
	\item I just found out that Pasiphae had sex with the white bull because she was cursed by Poseidon, according to fiction, but in the book she willingly wanted to have sex with the bull because she wanted to be known (was she also jealous?). State that this was an interesting change in story.
\end{itemize}

This short paper is all about a book called Circe, which is a fictional novel by Madeline Miller. Note that following paragraphs will not be a review about the book, but rather, it is going to be a verdict on the novel; I am going to write about what I really liked about the novel, and elaborate on it. Futhermore, I would also state my opinion about the difference between the characters that appear in ancient sources versus her retelling of it. Note that whenever I refer to historical or ancient references, I always refer to sources from the website theoi.com (\cite{theoi}). But long story short, I am genuinely fond of this novel; it kept me engaged until the very end. 

To start off, I would like to state what I truly appreciated in Miller's novel. Firstly, I loved how she was able to illustrate the uneasiness in the pact between the Titans and the Olympians (\cite{miller_circe_2018} 12 and 62), because it gave me a first person glance as to how the Titans and the Olympians felt about living in a time after the Titanomachy. Secondly, I was impressed how Miller presented Helios. He had a temper (\cite{miller_circe_2018} 3), enjoyed playing checkers (\cite{miller_circe_2018} 4) and looked down upon mortals (\cite{miller_circe_2018} 3). I also enjoyed reading about Circe's time spent with Helios, at first the author presented the interaction as a caring father spending time with one of his daughters like the fact that he let her ride in his chariot and let her see his herd of cattle (\cite{miller_circe_2018} 6), but as I read further into the book, he was just not at all caring. Ultimately, its the little things, the little attention to detail, that give the characters a breath of fresh air and in turn, give this book a charming flavour.

Now onto the star of the novel, Circe. The Circe that I was introduced in Miller's novel was a far cry from the Circe that I encountered in Homer's The Odyssey (\cite{homer_odyssey_1998}). True?I loved the interactions between Prometheus and Circe, because it humanizes her.I liked how they humanized Circe; how? By detailing how she felt with Glaucos, Prometheus etc. and compare it with her appearance in Circe

Another point I would like to add is that the two other standout characters, apart from Circe, are Daedalus and Pasiphae, in my opinion. First off, for Pasiphae, she was introduced very early on in the novel (\cite{miller_circe_2018} 5). She was shown to be an extremely unpleasant woman, always criticizing Circe every chance she gets, and with an absolute disregard for human life (\cite{miller_circe_2018} the one with Scylla). 

For Daedalus, he too was introduced very early on in the novel(\cite{miller_circe_2018} 28). The moment between Aeetes and 

In conclusion, as I have said before, I honestly enjoyed every second of this book. It was such a page-turner thanks to Miller's amazing ability to introduce and develop characters and give them a three-dimensional makeover of them and make them feel lifelike. Taking fictional characters and giving them In the end, you really feel for them. Personally, that alone is what I feel made the book such a winner. It had a lot of heart and it was emotional when it had to be. Despite the fact that the story was not original, as she adapted it from ancient and historical sources, the way she weaved the plot and presented it gave the novel a special charm to it. still had an amazing story, thanks to the fact that she was able to make virtually every single character in th Sure, the story was not original as the she adapted it from I give it 10/10.I loved how she took fictional characters and gave them heart and character, and really made it feel that they are actual people, even though some of them are Gods or Titans.

\newpage
\printbibliography
\end{document}

